\documentclass{article}

\usepackage{linear}
\usepackage{txfonts}

\usepackage{fullpage}

\setlength{\parindent}{0pt}

\begin{document}

\def\SP{\hspace{2em plus .1em minus 1em}}

\begin{center}\framebox{
  \begin{minipage}{0.99\textwidth}
    \textbf{\Large Rules and related definitions} \\
    Author: Kaustuv Chaudhuri \hfill \today
  \end{minipage}}
\end{center}

\subsection*{Rules}

\bgroup\small

\textbf{Structural}
\begin{gather*}
  \I[init]{". ; . ; A -->^0 A"}
  \SP
  \I[promote]{"YY' . ( . ; A ; . ) -->^w gg"}{"YY -->^w gg" & "YY' = YY -_s A"}
  \SP
  \I[copy]{"YY' . ( A ; . ; . ) -->^w gg"}{"YY -->^w gg" & "YY' = YY -_s A"}
\end{gather*}

\textbf{Multiplicative}
\begin{gather*}
  \I["tens_R"]{"YY_1 . YY_2 -->^{w_1 or w_2} A tens B"}
    {"YY_1 -->^{w_1} A" & "YY_2 -->^{w_2} B"}
  \\
  \I["tens_{L0}"]{"YY', A tens B -->^0 C"}
    {"YY -->^0 C" & "YY' = YY - A - B"}
  \SP
  \I["tens_{L1}"]{"YY', A tens B -->^1 gg"}
    {"YY -->^1 gg" & "YY' = YY - A -_? B"}
  \SP
  \I["tens_{L2}"]{"YY', A tens B -->^1 gg"}
    {"YY -->^1 gg" & "YY' = YY -_? A - B"}
  \\[1em]
  \Ic["one_R"]{". ; . ; . -->^0 one"}{}
  \SP
  \text{no "one_L"}
  \\[1em]
  \I["-o_{R0}"]{"YY' -->^0 A -o B"}{"YY -->^0 B" & "YY' = YY - A"}
  \SP
  \I["-o_{R1}"]{"YY' -->^1 A -o B"}{"YY -->^1 B" & "YY' = YY -_? A"}
  \SP
  \I["-o_{R2}"]{"YY' -->^1 A -o B"}{"YY -->^1 gg" & "YY' = YY - A" & "gg \subseteq B"}
  \\
  \I["-o_L"]{"( YY . YY'' ), A -o B -->^{w_1 or w_2} gg"}
    {"YY -->^{w_1} A" & "YY' -->^{w_2} gg" & "YY'' = YY' - B"}
\end{gather*}

\textbf{Additive}
\begin{gather*}
  \I["with_R"]{"YY_1 / w_1 + YY_2 / w_2 -->^{w_1 and w_2} A with B"}
    {"YY_1 -->^{w_1} A" & "YY_2 -->^{w_2} B"}
  \\
  \I["with_{L1}"]{"YY', A with B -->^w gg"}{"YY -->^w gg" & "YY' = YY - A" & "B \neq one"}
  \SP
  \I["with_{L2}"]{"YY', A with B -->^w gg"}{"YY -->^w gg" & "YY' = YY - B" & "A \neq one"}
  \\[1em]
  \Ic["top_R"]{". ; . ; . -->^1 top"}{}
  \SP
  \text{no "top_L"}
  \\[1em]
  \I["choice_{R1}"]{"YY -->^w A choice B"}{"YY -->^w A"}
  \SP
  \I["choice_{R2}"]{"YY -->^w A choice B"}{"YY -->^w B"}
  \\
  \I["choice_L"]{"( YY_1' / w_1 + YY_2' / w_2 ), A choice B -->^{w_1 and w_2} gg_1 \cup gg_2"}
    {"YY_1 -->^{w_1} gg_1" &
     "YY_1'" = \begin{cases}"YY_1" & "{{if }} A = one" \\ "YY_1 - A" & \text{otherwise}\end{cases} &
     "YY_2 -->^{w_2} gg_2" &
     "YY_2'" = \begin{cases}"YY_2" & "{{if }} B = one" \\ "YY_2 - B" & \text{otherwise}\end{cases}
    }
  \\[1em]
  \text{no "zero_R"}
  \SP
  \Ic["zero_L"]{". ; . ; zero -->^1 ."}{}
\end{gather*}

\textbf{Exponential}
\begin{gather*}
  \Ic["!_R"]{"GG ; . ; . -->^0 {! A}"}{"GG ; . ; . -->^w A"}
  \SP
  \text{no "!_L"}
\end{gather*}

\egroup


\subsection*{Extraction}
\label{sec:extraction}

There are three kinds of extraction -- strict ("-_s"), normal ("-") and lax ("-_?").

\paragraph{Strict extraction}
\label{sec:strict-extraction}

\begin{align*}
  \left.
    \begin{array}{r}
      "( GG, A ; PP ; DD ) -_s {!A}" \\
      "( GG ; PP, A ; DD ) -_s A with one" \\
      "( GG ; PP, A ; DD ) -_s one with A" \\
      "( GG ; PP ; DD, A ) -_s A"
    \end{array}
  \right\} &= "GG ; PP ; DD" \\
\end{align*}

\paragraph{Normal extraction}
\label{sec:normal-extraction}

\begin{align*}
  \left.
    \begin{array}{r}
      "( GG, A ; PP ; DD ) - {!A}" \\
      "{{if }} A \notin GG {{ then }} ( GG ; PP ; DD ) - {!A}" \\
      "( GG ; PP, A ; DD ) - A with one" \\
      "{{if }} A \notin PP {{ then }} ( GG ; PP ; DD ) - A with one" \\
      "( GG ; PP, A ; DD ) - one with A" \\
      "{{if }} A \notin PP {{ then }} ( GG ; PP ; DD ) - one with A" \\
      "( GG ; PP ; DD, A ) - A"
    \end{array}
  \right\} &= "GG ; PP ; DD" \\
\end{align*}

\paragraph{Lax extraction}
\label{sec:lax-extraction}

is like normal extraction, with one additional case:
\begin{align*}
  \left.
    \begin{array}{r}
      "{{if }} A \notin DD {{ then }} ( GG ; PP ; DD ) -_? A"
    \end{array}
  \right\} &= "GG ; PP ; DD" \\
\end{align*}


\subsection*{Affixion}
\label{sec:affixion}

There are two ways to affix contexts to each other -- multiplicatively
or additively.


\paragraph{Multiplicative}
\label{sec:multiplicative}

affixion is total, with the following simple rule:
\begin{gather*}
  "( GG ; PP ; DD ) . ( GG' ; PP' ; DD' )\ =\ GG, GG' ; PP, PP' ; DD, DD'"
\end{gather*}
A slight abbreviation: "( GG ; PP ; DD ), A\ = \ GG ; PP ; DD, A"

\paragraph{Additive}
\label{sec:additive}

\sdef{lub}{\mathop{\text{lub}}}

affixion is partial and depends on weakening annotations.
\begin{gather*}
  "( GG_1 ; PP_1 ; DD_1 ) / w_1 + ( GG_2 ; PP_2 ; DD_2 ) / w_2"\ =\
  "GG_1, GG_2 ; PP_1 / 1 + PP_2 / 1 ; DD_1 / w_1 + DD_2 / w_2"
\end{gather*}
where the affixion of multisets is defined as follows:
\begin{gather*}
  "DD_1 / w_1 + DD_2 / w_2"\ = \
  \begin{cases}
    "lub ( DD_1, DD_2 )" & "{{if }} w_1 = w_2 = 1" \\
    "DD_1" & "{{if }} w_1 = 1 and w_2 = 0 and DD_1 \subseteq DD_2" \\
    "DD_2" & "{{if }} w_2 = 1 and w_1 = 0 and DD_2 \subseteq DD_1" \\
    "DD_1" & "{{if }} w_1 = w_2 = 0 and DD_1 = DD_2"
  \end{cases}
\end{gather*}
and "lub ( DD_1, DD_2 )" is the least multiset to include both "DD_1"
and "DD_2".\footnote{This particular operation will be more complex in
  the first-order case.}

\end{document}
